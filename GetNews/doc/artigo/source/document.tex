\documentclass[12pt]{article}
\usepackage{sbc-template}
\usepackage{graphicx,url}
\usepackage{alltt,fancyvrb,algorithm,algorithmic}
\usepackage[brazil]{babel}   
\usepackage[latin1]{inputenc}  
\usepackage{booktabs}
\sloppy

\title{An�lise, S�ntese e Recupera��o de Not�cias publicadas na Internet}

\author{Fabr�cio J. Barth}

\address{Qualquer Institui��o
\email{fabricio.barth@gmail.com}
}

\begin{document}

\maketitle

\begin{resumo}
Este trabalho\ldots
\end{resumo}

\section{Introdu��o}

O objetivo deste projeto � a implementa��o de um mecanismo respons�vel por
monitorar e armazenar todas as not�cias publicadas por algumas fontes jornal�sticas presentes na Internet. 

O monitoramento e armazenamento das not�cias dever�o ser �teis para:

\begin{itemize}	
  \item Medir a quantidade de not�cias produzidas por fonte, hor�rio e dia da
  semana, descobrindo se existe algum
  padr�o na quantidade de not�cias produzidas pelas fontes jornal�sticas, por exemplo, se dado um evento 
  importante ser� que o padr�o de not�cias produzidas � alterado? 
  \item Criar um �ndece �nico de not�cias produzidas na Internet. 
\end{itemize} 

\bibliographystyle{sbc}
\bibliography{atech,doutorado}

\end{document}